\documentclass[sigconf]{acmart}


\setcopyright{none}


\begin{document}

\title{Handwriting Recognition with SQL\\\hspace{6.88cm}\footnotesize and a tiny bit of web stuff}
\author{Noah Doersing TODO remove topmargin}
\affiliation{}
\email{noah.doersing@student.uni-tuebingen.de}

\maketitle

\section*{Abstract}
This paper provides a sample of a \LaTeX\ document which conforms, somewhat loosely, to the formatting guidelines for ACM SIG Proceedings.\footnote{This is an abstract footnote}

benjamin's idea worse because
* 4-direction patterns in the first lookup table that map to a *single* character essentially skip the second lookup table (see `INSERT` in line 116 of the setup file). This could be somehow emulated using a query (if `COUNT(candidate\_character)` for the current pattern is 1, skip the feature-matching step), but that adds cruft to the query.
* Deciding between 7 and 1 requires features F, while deciding between 7 and 3 requires looking at features G != F. Could also be patched over at the query level.
* More verbose first lookup table, slimmer second lookup table.
* Second lookup table does not anymore indicate which 4-direction pattern a given feature match belongs to, making debugging/extending harder.
* Effectively requires ranking results based on closeness of match, e.g. number of features used (or VERY careful design of the second lookup table, which I didn't have time for before my presentation).
* Conceptually further from original tree structure, which can be a bad or good thing.

\section{Introduction}

The main content goes here. testign

\section{Example/Demo?}

TODO really here? or something else?

\section{Implementation}

\subsection{Smoothing and Thinning}

test

\subsection{Curvature computation}

test

\subsection{Corner detection}

test

\subsection{Additional feature extraction}

test

\subsection{Mapping TODO}

test

\section{Evaluation}

test

\section{Related/Future work}

test

TODO conclusion, bibliography: essay, memo, memo on rand tablet, video of alan kay



\end{document}
